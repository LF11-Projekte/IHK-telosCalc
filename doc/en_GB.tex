\documentclass[
a4paper,
12pt
]{scrartcl}

\usepackage[utf8]{inputenc}
\usepackage[T1]{fontenc}
\usepackage[oldstylenums,largesmallcaps,nomath]{kpfonts}

\usepackage[english]{babel}
\usepackage[german=guillemets]{csquotes}
\usepackage{xpatch}
\usepackage{hyphenat}

\usepackage{graphicx}
\usepackage{subcaption}
\usepackage{microtype}
\usepackage{geometry}
\usepackage{caption}
\usepackage{float}
\usepackage{framed}
\usepackage{xcolor}
\usepackage{hyperref}
\usepackage{enumitem}
\usepackage{listings}
\usepackage{booktabs}
\usepackage{soul}

\geometry{margin=2.5cm}
\hypersetup{
	colorlinks=true,
	linkcolor=black,
	urlcolor=blue,
	pdftitle={telosCalc — User Documentation},
	pdfauthor={telosCalc Team}
}

% Colors & Styles
\definecolor{accent}{HTML}{2A4B7C}
\definecolor{hintbg}{HTML}{F2F7FB}
\definecolor{hintborder}{HTML}{D6E6F4}
\newenvironment{hintbox}{\begin{framed}\noindent\color{black}\setlength{\FrameSep}{8pt}\setlength{\FrameRule}{0.6pt}\definecolor{shadecolor}{named}{hintbg}}{\end{framed}}

\subject{\so{User Documentation}}
\subtitle{ -- \\ The \textsc{IHK} Grade Calculation Tool for IT Specialists}
\title{\Huge \textsc{telosCalc}}
\author{Karl Jahn \and Damian Carstens \and Kai Weißenborn}
\date{\today}

\begin{document}
	\maketitle
	
	\begin{center}
		\textbf{Note:} This document has been \textbf{automatically translated from German. Minor inaccuracies may occur.}
	\end{center}
	
	\bigskip
	
	\begin{center}
		\textbf{Short Description:} \textsc{telosCalc} is a tool for calculating the \textsc{IHK} final exam grades of IT specialists in the field of application development. It is primarily a desktop application for Windows, but experimental support for \texttt{macOS} and \texttt{Linux} systems is available. This manual is intended for end users and contains installation, usage, and troubleshooting instructions.
	\end{center}
	
	\bigskip
	\tableofcontents
	\newpage
	
	\section{System Requirements}
	\begin{itemize}
		\item Windows 10 or newer (for the provided \texttt{telosCalc.exe})
		\item Optional: Python 3.8+ if running from source or using \texttt{Linux} / \texttt{macOS}.
	\end{itemize}
	
	\section{Installation \& Launch}
	\subsection*{Running the EXE}
	\begin{enumerate}
		\item Copy \texttt{telosCalc.exe} to any directory.
		\item Double-click to start. On first launch, a configuration file (\texttt{telosCalc.conf}) will be created in the same directory. This file is read each time the program starts, restoring the settings from the previous session. When closing the application, the configuration file is automatically updated with the current settings.
	\end{enumerate}
	
	\subsection*{Running from Source (Optional)}
	\begin{enumerate}
		\item Install Python 3.8 or newer.
		\item Create a virtual environment: \texttt{python -m venv .venv} and activate it via\\ \texttt{.\textbackslash.venv\textbackslash{}Scripts\textbackslash{}activate}.
		\item Install dependencies: \texttt{pip install -r requirements.txt}
		\item \underline{Run the application:} \texttt{python main.py}\\[.65em]\emph{or}:\\[.65em] \underline{Build the application:}
		Using \texttt{pip install pyinstaller \&\& pyinstaller main.spec}. You can specify the target architecture in \texttt{main.spec}; if none is set, the current system architecture is used.
	\end{enumerate}
	
	\section{Using the GUI}
	
	The menu bar (see Figure~\ref{fig:menu}) provides the options \texttt{File}, \texttt{Language}, \texttt{Appearance}, and \texttt{Help}. Under \texttt{Help}, the options \texttt{Documentation} (linking to this document) and \texttt{About} (showing application info) are available.
	
	\begin{figure}[H]
		\centering
		\fbox{\includegraphics[width=0.75\linewidth]{img/menu.png}}
		\caption{Menu bar / Import \& Export of input parameters}
		\label{fig:menu}
	\end{figure}
	
	\subsection{Color Scheme (\texttt{Appearance})}
	The interface offers multiple color schemes, each available in light and dark variants. These can be selected via \texttt{Menu Bar > Appearance > \{light / dark\} > \{color\}} (see Figure~\ref{fig:color}). Figure~\ref{fig:dark} shows an example of a dark color scheme.
	
	\begin{figure}[H]
		\centering
		\begin{minipage}{.5\textwidth}
			\centering
			\fbox{\includegraphics[width=0.75\linewidth]{img/light-modes.png}}
			\captionof{figure}{Available Appearance Modes}
			\label{fig:color}
		\end{minipage}%	
		\begin{minipage}{.5\textwidth}
			\centering
			\fbox{\includegraphics[width=0.75\linewidth]{img/lang-dark.png}}
			\captionof{figure}{Dark Theme / Language Options}
			\label{fig:dark}
		\end{minipage}
	\end{figure}
	
	Under \texttt{Language}, you can switch between the German and English versions of the application (see Figure \ref{fig:dark}). The checkbox next to the language indicates the active one. The same menu also provides the \texttt{Speech Output} option — see Section \ref{sprachausg}.
	
	\subsection{Input Parameters}
	Input parameters can be saved or loaded via \texttt{File > \{Load Input Parameters / Save Input Parameters\}} (see Figure~\ref{fig:menu}). All entered data from the input fields is stored in \texttt{JSON} format. Loading parameters overwrites any existing input values.
	
	\subsection{Speech Output \& Accessibility} \label{sprachausg}
	When speech output is enabled, the interface provides spoken feedback for navigation. Hovering over elements or navigating with the \texttt{Tab} key triggers speech output. The \texttt{Announce Grades} button also becomes active; clicking it announces all partial grades, the overall average, and whether the exam was passed, in the selected target language.
	
	All interactive elements can be navigated using the \texttt{Tab} key. The top menu is accessed with \texttt{[Alt]}, navigated via arrow keys, space, and \texttt{Tab}, and confirmed with \texttt{[ENTER]}. The \texttt{[ESC]} key exits the top menu.
	
	\section{Grade Calculation}
	
	The grade calculation follows the official IHK criteria as outlined in the training regulations, BBiG, and IHK guidelines. Specifically, the decision process for “passed” or “not passed” follows the logic shown below:\\[1em]
	
	\includegraphics[width=\textwidth]{img/criteria.png}\\[.5em]
	
	The option for selecting an oral supplementary exam only becomes available if such an exam is applicable. You can then navigate to the corresponding element with \texttt{Tab} or click the arrow to open the dropdown menu (see Figure~\ref{fig:mep}). Only valid subjects are selectable. Once a valid subject is chosen, a corresponding score can be entered.
	
	\begin{figure}[h]
		\centering
		\fbox{\includegraphics[width=0.55\linewidth]{img/mep.png}}
		\caption{Oral Supplementary Exam}
		\label{fig:mep}
	\end{figure}
	
	\section{Troubleshooting}
	\begin{description}
		\item[Application won't start:] Check antivirus / SmartScreen; try running as administrator.
		\item[Configuration error:] Rename or delete \texttt{telosCalc.conf}; restart the application if necessary.
		\item[TTS not working:] Run from source; check console messages and install any missing packages.
	\end{description}
	
\end{document}
